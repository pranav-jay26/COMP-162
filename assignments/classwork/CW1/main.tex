\documentclass[12pt]{article}


\usepackage[utf8]{inputenc}
\usepackage{latexsym,amsfonts,amssymb,amsthm,amsmath,graphicx}

\setlength{\parindent}{0in}
\setlength{\oddsidemargin}{0in}
\setlength{\textwidth}{6.5in}
\setlength{\textheight}{8.8in}
\setlength{\topmargin}{0in}
\setlength{\headheight}{18pt}



\title{Comp 162 Classwork 1}
\author{Pranav Jayakumar}

\begin{document}
\maketitle

\subsection*{Question 1}
What do you think Data Analytics means?\\ 

\textbf{Solution}\\ 

Data Analytics means using computer programs and mathematical models to extract useful insights about data. 
\vspace{0.25in}
\subsection*{Question 2}
Which is the data analyst and which is the data scientist in the following scenario?\\ 

\begin{itemize}
  \item Job 1: Determine which categories of games are popular among each demographic
  \item Job 2: Develop a model to predict which games that are about to be release will be the most successful
\end{itemize} 

\textbf{Solution}\\ 

In the first scenario, the person performs the job of a \textbf{data analyst}. This is because the person is only answering a specific question about the data.\\ 

In the second scenario, the person performs the job of a \textbf{data scientist}. This is because the person is using the data collected to make a prediction. \\ 
\vspace{0.25in}
\subsection*{Question 3}
Where else do you see analytics in your own life? What personal data analytics do you have access to?\\ 

\textbf{Solution}\\ 
We see data analytics in virtually everything around us. Some key examples include Spotify wrapped, a personal health app, banking information, and many more.
\vspace{0.25in}
\subsection*{Question 4}
What are the variables in the "Welcome Survey" data? What are the observations?\\ 
The variables of the "Welcome Survey" data are as follows:
\begin{itemize}
  \item Year
  \item Excitement
  \item Height 
  \item Season 
  \item Introvert vs. Extrovert 
  \item Handedness
  \item Sibling
  \item Bedtime 
  \item Caffeine
  \item Tennis Ball Color
  \item Dress Color 
\end{itemize}

\textbf{Solution}\\ 
\\ 
We observe that students who began college in 2024 and those who began in 2021 exhibit the highest levels of excitement. 
We also observe that the median height of the students in the survey is 67.5 inches. Finally, we observe that the class 
is mostly introverted, with there being 30 introverts and 4 extroverts out of the 34 students. 
\vspace{0.25in}
\subsection*{Question 5}
Classify all of the in the "Welcome Survey" data into the following categories:
\begin{enumerate}
  \item Binary (categorical variable with two choices)
  \item Nominal (categorical variable with more than two choices)
  \item Ordinal (categorical variable with intrinsic ordering)
  \item Continuous (quantitative variable that can take an infinite number of values)
  \item Discrete (quantitative variable that can only take a fixed number of values)
\end{enumerate}

\textbf{Solution}\\ 
\\ 
The classification of each variable is as follows:\\ 
\begin{enumerate}
  \item Binary
    \begin{itemize}
      \item Introvert vs Extrovert
      \item Tennis Ball Color
    \end{itemize}
  \item Nominal
    \begin{itemize}
      \item Season
      \item Handedness
      \item Dress Color
    \end{itemize}
  \item Ordinal
    \begin{itemize}
      \item Excitement
    \end{itemize}
  \item Continuous
    \begin{itemize}
      \item Height
      \item Bedtime
      \item Caffeine
    \end{itemize}
  \item Discrete
    \begin{itemize}
      \item Year
      \item Sibling
    \end{itemize}
\end{enumerate}
\vspace{0.25in}
\subsection*{Question 6}
What question do you have about the "Welcome Survey" data that can be answered using a 1-way or 2-way frequency table?\\ 

\textbf{Solution}\\ 
\\ 
We can answer the question "How many individuals percieve the dress as being blue and black versus being white and gold?" using a 1-way frequency table. 
\vspace{0.25in}
\end{document}
